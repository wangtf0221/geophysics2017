\section{Local dip filtering}
Standard depth tomography methods often do not converge, converge slowly, or
converge to a model that is geologically unrealistic. By using dip based steering
filters rather than standard isotropic regularization operators, convergence speed
and final model quality are improved. By formulating the tomography operator in
two-way vertical traveltime (τ,ξ) rather than in depth (z,x) coordinates, we can
uncouple velocity estimation from reflector position estimation. In this coordinate
system we avoid many of the problems associated with the depth tomography
problem, and converge quickly to a reasonable solution.

Depth velocity estimation is one of the most difficult problems in reflection
seismology.
The tomography problem is by nature non-linear and under-determined. A
standard technique is to perform a non-linear loop over a linearized inversion
problem.
Unfortunately, such a technique is prone to converge to a local minimum of the
objective function (Phillips and Fehler, 1991). In addition, each of the linearized
inversion
problems are under-determined, requiring some type of regularization. Often
an isotropic operator is used to regularize the tomography problem (Biondi, 1990),
but these operators tend to fill-in the null space with isotropic features that are
often
geologically unreasonable (Etgen, 1997).

Fortunately, we often have other sources of information, such as well logs, stacked
sections, or geologist’s interpretation, that we can use to construct anisotropic
operators
(steering filters) (Clapp et al., 1997, 1998) that fill the null space with more
geologically reasonable velocities. Convergence speed can be improved by changing
from a regularized to a preconditioned problem (Claerbout and Nichols, 1993). By
forming the regularization operator in a helical coordinate system we can efficiently
obtain an inverse operator by polynomial division (Claerbout, 1998). This new operator
can be used as a preconditioner, creating an equivalent optimization problem
(Fomel et al., 1997) that converges significantly faster

Smoothing is performed by solving a sparse symmetric positive-definite
system of equations: (S'S+D'TD)y = S'Sx, where D is a matrix of 
derivative (difference) operators, S is a matrix of smoothing (sum)
operators, T is a matrix of tensor filter coefficients, x is an input 
image and y is an output image.

