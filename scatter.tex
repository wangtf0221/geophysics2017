\subsection{Scattered wavefields of one single scatter}
The radiation patterns we plot are not the wavefield energy scattered by
the perturbations.However,analyze the scattered wavefields using wave mode separation 
can help us to have a insight of the real scattered wavefields.So, we can compare the
theoretical result (equation 19) with the real simulated perturbed wavefields.
In anisotropic media, the parameterization of stiffness will simplify the test of our
theory.We calculate the theory radiation patter with equation:
\begin{equation}
	\delta
	u^{MN}_n(\mathbf{x}_r,\omega)=\int\limits_{V(\mathbf{x}^{'})}
	A(\omega)\Omega^{MN}(c_{ijkl})\delta c_{ijkl}dV(\mathbf{x}^{'}),
	\label{eq:Kernel}
\end{equation}
where
\begin{equation}
	A(\omega)=\sum_{m=1,3}f_m(\mathbf{x}_s,\omega)g^M_{m,s}g^N_{n,r}
	G^M_sG^N_r\frac{\omega^2}{v^M_sv^N_r}.
	\label{eq:A}
\end{equation}
Be ware of the difference from equation 19.Here we intent to calculate the whole
scattered wavefield, so there are two incident wavefield (x- and z- component) that
will cause the scattering phenomenon.That's why there is a summation in the equation.
To test our theory, we design a $360^\circ$
aquisition geometry with a single scatter in the center of the geometry.And the incident
angle (source location) varies from $0^\circ$ to $90^\circ$ with a increment of
$30^\circ$.There is no need to test incident angle of $90^\circ$ to $180^\circ$
because of symmetry.We solve the elastic velocity-stress equation with the
staggered-grid finite difference scheme to get the simulated wavefields.
The scattered wavefield is obtained by subtracting the background wavefield from the
perturbed wavefield.

\inputdir{scatter}
\multiplot*{1}{real.epsi.x,theory.epsi.x}{width=1.0\textwidth}
{The normalized x-component of scattered wavefield energy in terms of $c_{11}$ perturbation with
different incident angle.In this two pictures, red line represents PS mode and blue line is PP
	mode.(a):results from simulation;(b):theoretical results from this paper.
	}

\inputdir{scatter}
\multiplot*{1}{real.epsi.y,theory.epsi.y}{width=1.0\textwidth}
{The normalized y-component of scattered wavefield energy in terms of $c_{11}$ perturbation with
different incident angle.In this two pictures, red line represents PS mode and blue line is PP
	mode.(a):results from simulation;(b):theoretical results from this paper.
	}

\inputdir{scatter}
\multiplot*{1}{real.delta.x,theory.delta.x}{width=1.0\textwidth}
{The normalized x-component of scattered wavefield energy in terms of $c_{13}$ perturbation with
different incident angle.In this two pictures, red line represents PS mode and blue line is PP
	mode.(a):results from simulation;(b):theoretical results from this paper.
	}

\inputdir{scatter}
\multiplot*{1}{real.delta.y,theory.delta.y}{width=1.0\textwidth}
{The normalized y-component of scattered wavefield energy in terms of $c_{13}$ perturbation with
different incident angle.In this two pictures, red line represents PS mode and blue line is PP
	mode.(a):results from simulation;(b):theoretical results from this paper.
	}

\inputdir{scatter}
\multiplot*{1}{real.vp0.x,theory.vp0.x}{width=1.0\textwidth}
{The normalized x-component of scattered wavefield energy in terms of $c_{33}$ perturbation with
different incident angle.In this two pictures, red line represents PS mode and blue line is PP
	mode.(a):results from simulation;(b):theoretical results from this paper.
	}

\inputdir{scatter}
\multiplot*{1}{real.vp0.y,theory.vp0.y}{width=1.0\textwidth}
{The normalized y-component of scattered wavefield energy in terms of $c_{33}$ perturbation with
different incident angle.In this two pictures, red line represents PS mode and blue line is PP
	mode.(a):results from simulation;(b):theoretical results from this paper.
	}

\inputdir{scatter}
\multiplot*{1}{real.vs0.x,theory.vs0.x}{width=1.0\textwidth}
{The normalized x-component of scattered wavefield energy in terms of $c_{55}$ perturbation with
different incident angle.In this two pictures, red line represents PS mode and blue line is PP
	mode.(a):results from simulation;(b):theoretical results from this paper.
	}

\inputdir{scatter}
\multiplot*{1}{real.vs0.y,theory.vs0.y}{width=1.0\textwidth}
{The normalized y-component of scattered wavefield energy in terms of $c_{55}$ perturbation with
different incident angle.In this two pictures, red line represents PS mode and blue line is PP
	mode.(a):results from simulation;(b):theoretical results from this paper.
	}

\inputdir{scatter}
\multiplot*{1}{theory.PP.x,theory.PP.y}{width=1.0\textwidth}
{The normalized scattered wavefield energy in terms of PP wave mode with
different incident angle for x- anx z- component.In this two pictures, lines with
different color represent different parameter perturbation, blue($\delta c_{13}$),
magenta($\delta c_{11}$),red($\delta c_{33}$), black($\delta c_{55}$).
	}

\inputdir{scatter}
\multiplot*{1}{theory.PS.x,theory.PS.y}{width=1.0\textwidth}
{The normalized scattered wavefield energy in terms of PS wave mode with
different incident angle for x- anx z- component.In this two pictures, lines with
different color represent different parameter perturbation, blue($\delta c_{13}$),
magenta($\delta c_{11}$),red($\delta c_{33}$), black($\delta c_{55}$).
	}

\inputdir{scatter}
\multiplot*{1}{theory.SP.x,theory.SP.y}{width=1.0\textwidth}
{The normalized scattered wavefield energy in terms of PS wave mode with
different incident angle for x- anx z- component.In this two pictures, lines with
different color represent different parameter perturbation, blue($\delta c_{13}$),
magenta($\delta c_{11}$),red($\delta c_{33}$), black($\delta c_{55}$).
	}

\inputdir{scatter}
\multiplot*{1}{theory.SS.x,theory.SS.y}{width=1.0\textwidth}
{The normalized scattered wavefield energy in terms of PS wave mode with
different incident angle for x- anx z- component.In this two pictures, lines with
different color represent different parameter perturbation, blue($\delta c_{13}$),
magenta($\delta c_{11}$),red($\delta c_{33}$), black($\delta c_{55}$).
	}

\inputdir{scatter}
\multiplot*{1}{real.c11c33c13.x,real.c11c33c13.y}{width=1.0\textwidth}
{The normalized scattered wavefield energy in terms of c11, c33 and c13 with
different incident angle for x- anx z- component.In this two pictures, lines with
different color represent different wave modes, blue(PP),
red(PS).
	}

\inputdir{scatter}
\multiplot*{1}{real.c11c33c13c55.x,real.c11c33c13c55.y}{width=1.0\textwidth}
{The normalized scattered wavefield energy in terms of c11, c33, c55 and c13 with
different incident angle for x- anx z- component.In this two pictures, lines with
different color represent different wave modes, blue(PP),
red(PS).
	}

\inputdir{Fig.theory.iso}
\multiplot*{1}{lamdaX,lamdaY}{width=1.0\textwidth}
{The theoritical x- and z-component of scattered wavefield energy in terms of $\lambda$ perturbation with
different incident angle.In this two pictures, red line represents PP mode and blue line is PS
	mode.(a):x-component;(b):z-component.
	}

\inputdir{Fig.theory.iso}
\multiplot*{1}{muX,muY}{width=1.0\textwidth}
{The theoritical x- and z-component of scattered wavefield energy in terms of $\mu$ perturbation with
different incident angle.In this two pictures, red line represents PP mode and blue line is PS
	mode.(a):x-component;(b):z-component.
	}
