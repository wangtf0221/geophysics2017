\inputdir{Fig.transmitted.wave}
\multiplot*{2}{Original.transmitted,Separated.transmitted}{width=0.4\textwidth}
{Radiation patterns in terms of different parameter perturbations for transmition
	geometry.In this two pictures, red line:$\Omega (V_{hor})$,green line:$\Omega
	(V_{P0})$ and blue line:$\Omega (V_{nmo})$.(a):the original radiation pattern of
	mixed wavefields;(b)different combinations of wave modes.$\Omega (V_{hor})=
  \Omega^{PP}(V_{hor})+\Omega^{PS}(V_{hor})+\Omega^{SP}(V_{hor})+\Omega^{SS}(V_{hor})$;
  $\Omega (V_{P0})=\Omega^{PP}(V_{P0})-\Omega^{SS}(V_{P0})$;
  $\Omega (V_{nmo})=-\Omega^{PP}(V_{nmo})+\Omega^{SP}(V_{nmo})+\Omega^{PS}(V_{nmo})+
  \Omega^{SS}(V_{nmo})$.}

\inputdir{Fig.reflected.wave}
\multiplot*{2}{Original.Reflection,Reflection}{width=0.4\textwidth}
{Radiation patterns in terms of different parameter perturbations for reflection
	geometry.In this two pictures, red line:$\Omega (V_{hor})$,green line:$\Omega
	(V_{P0})$ and blue line:$\Omega (V_{nmo})$.(a):the original radiation pattern of
	mixed wavefields;(b)different combinations of wave modes.$\Omega (V_{hor})=
  \Omega^{PP}(V_{hor})+\Omega^{PS}(V_{hor})+\Omega^{SP}(V_{hor})+\Omega^{SS}(V_{hor})$;
  $\Omega (V_{P0})=\Omega^{PP}(V_{P0})-\Omega^{SS}(V_{P0})$;
  $\Omega (V_{nmo})=-\Omega^{PP}(V_{nmo})+\Omega^{SP}(V_{nmo})+\Omega^{PS}(V_{nmo})+
  \Omega^{SS}(V_{nmo})$.}

\inputdir{scatter}
\multiplot*{1}{real.epsi.x,theory.epsi.x}{width=1.0\textwidth}
{The normalized x-component of scattered wavefield energy in terms of $c_{11}$ perturbation with
different incident angle.In this two pictures, red line represents PS mode and blue line is PP
	mode.(a):results from simulation;(b):theoretical results from this paper.
	}

\inputdir{scatter}
\multiplot*{1}{real.epsi.y,theory.epsi.y}{width=1.0\textwidth}
{The normalized y-component of scattered wavefield energy in terms of $c_{11}$ perturbation with
different incident angle.In this two pictures, red line represents PS mode and blue line is PP
	mode.(a):results from simulation;(b):theoretical results from this paper.
	}

\inputdir{scatter}
\multiplot*{1}{real.delta.x,theory.delta.x}{width=1.0\textwidth}
{The normalized x-component of scattered wavefield energy in terms of $c_{13}$ perturbation with
different incident angle.In this two pictures, red line represents PS mode and blue line is PP
	mode.(a):results from simulation;(b):theoretical results from this paper.
	}

\inputdir{scatter}
\multiplot*{1}{real.delta.y,theory.delta.y}{width=1.0\textwidth}
{The normalized y-component of scattered wavefield energy in terms of $c_{13}$ perturbation with
different incident angle.In this two pictures, red line represents PS mode and blue line is PP
	mode.(a):results from simulation;(b):theoretical results from this paper.
	}

\inputdir{scatter}
\multiplot*{1}{real.vp0.x,theory.vp0.x}{width=1.0\textwidth}
{The normalized x-component of scattered wavefield energy in terms of $c_{33}$ perturbation with
different incident angle.In this two pictures, red line represents PS mode and blue line is PP
	mode.(a):results from simulation;(b):theoretical results from this paper.
	}

\inputdir{scatter}
\multiplot*{1}{real.vp0.y,theory.vp0.y}{width=1.0\textwidth}
{The normalized y-component of scattered wavefield energy in terms of $c_{33}$ perturbation with
different incident angle.In this two pictures, red line represents PS mode and blue line is PP
	mode.(a):results from simulation;(b):theoretical results from this paper.
	}

\inputdir{scatter}
\multiplot*{1}{real.vs0.x,theory.vs0.x}{width=1.0\textwidth}
{The normalized x-component of scattered wavefield energy in terms of $c_{55}$ perturbation with
different incident angle.In this two pictures, red line represents PS mode and blue line is PP
	mode.(a):results from simulation;(b):theoretical results from this paper.
	}

\inputdir{scatter}
\multiplot*{1}{real.vs0.y,theory.vs0.y}{width=1.0\textwidth}
{The normalized y-component of scattered wavefield energy in terms of $c_{55}$ perturbation with
different incident angle.In this two pictures, red line represents PS mode and blue line is PP
	mode.(a):results from simulation;(b):theoretical results from this paper.
	}
