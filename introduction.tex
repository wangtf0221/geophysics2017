\section{Introduction}
With the emergence of long-offset wide-azimuth acquisitions and broad-band sources,
full waveform inversion (FWI) has been recognized as an
efficient tool for constructing velocity models and quantitative seismic imaging, see
\cite{virieux2009overview} for a review.
Though the acoustic FWI, primarily focusing on P-wave velocity inversion, has been
widely studied in the past decades
\cite[]{tarantola1984,pratt1998gauss,shipp:2002}, people are paying more attention
to the waveform inversion under the elastic assumption, referred to as elastic full waveform inversion
(EFWI) \cite[]{tarantola1986}.
Waveform inversion provides high-resolution model estimation of the elastic
properties, but it
suffers from the cycle-skipping easily because of its insensitivity to the low and intermediate wavenumber components of the model
when the acquisition illumination is poor and/or good initial models are unavailable during the inversion\cite[]{sears2008,brossier2009}. 
Besides, multi-parameter trade-off effects and more complicated elastic wave phenomena 
will further increase the difficulties for EFWI.
To deal with the nonlinearities and parameter trade-offs, more
preconditioning, hierarchical strategies and parameterization investigation should be
considered during EFWI 
\cite[]{sears2008,prieux:2013b,operto2013guided,wang:2015,Oh2016}.

%In the surface observation geometry, full waveform inversion (FWI) often suffers from the
%local minimal due to the absence of low frequencies in the data when the low-wavenumber
%component is missing in the initial model.
For the classical FWI, the long-offset data
corresponding to diving waves are very important to build the long-to-intermediate
wavelengths of the model. However, the penetration depths of
diving waves are always far from sufficient to reach the target in the deeper part
%with the limited surface offset,
even using the wide-aperture surveys. 
In addition, the low signal-to-noise ratio at the far
offset is also a limit for the classical FWI relying on the diving waves. 
Therefore, people tried to utilize the reflections
to help build the macro-model containing low-to-intermediate wavenumber
in the deep part\cite[]{Stork1992,ChaventEtAl1994,clement:2001,Symes2008a,xu:2012}.
This process can be implemented in the image domain or the data domain. 
Actually, the image-domain ray-based tomography
\cite[]{Stork1992,Woodward1992,Woodward2008,Jones2010} has already been a standard workflow to obtain the background velocity by flatting the common
image gathers. But when the lateral velocity variation is strong, the ray-based method would fail to
present the wave propagation underground.
%, which will probably lead to the failure of tomography.  
To overcome the limit of ray theory, 
people have made great efforts to develop the approaches of reflection inversion that employ
waveform or traveltime information based on wave-equation
\cite[]{xu:2012,Ma2013,Wu2015b,Zhou2015,Wang2015,Chi2015}.

The misfit function of reflection inversion can be built in image domain in the manner of wave equation migration
velocity analysis (WEMVA), which tries to maximize the
energy at zero-offset location in the extended subsurface image gathers
\cite[]{Symes2008,Almomin2012,SunEtAl2012,biondi2013}.
%Alternatively, the image domain strategy also can be fulfilled by the wave equation
%migration velocity analysis (WEMVA), in which velocity is updated
%through minimizing the energy of non-zero offset on the extended-domain gathers \cite[]{Symes, 2008; Fleury
%and Perrone, 2012; Biondi and Almomin, 2013}.
\cite{RaknesEtAl2016} 
developed an image-domain method to invert P-wave velocity ($V_p$) in the 3D elastic media. 
\cite{Wang2017WEMVA} proposed to employ the extended PS image in WEMVA to update the
S-wave velocity ($V_s$) with the help of elastic wave mode decomposition.
%But it is a big issue that the extended domain approach requires huge computational cost,
%especially in 3D case.
However, the extended-domain methods are limited in filed data due to its prohibitively computational cost,
especially in 3D case.
While in the data domain,
\cite{xu:2012} suggested using a reflection waveform inversion (RWI) method to suppress the
nonlinearity in FWI, 
which aim to invert the long-wavelength components of the model by using the reflections 
predicted by migration/demigration process.
Recently, \cite{Zhou2015} proposed a joint FWI method that combines the diving and reflected waves to
utilize both RWI and the conventional FWI.
\cite{Wu2015b} developed an RWI scheme to simultaneously invert the background velocity and the
perturbation in acoustic media,  and recently extended to elastic case by
\cite{Guo2016}.
RWI highly relies on the accurate reflectivity to generate the reflections that can match the
observed data. However, it is very challenging and also expensive to obtain a good reflectivity model through 
least-square migration when initial model is far away from the true one.

%In addition, \cite{Wang:2015} found that wave mode decomposition may be beneficial to deal with the
%elastic parameter trade-offs.
Compared with the waveform information, 
%traveltime is more sensitive and linearly related to
%low-wavenumber model perturbation. Using traveltime inversion will be more robust and helpful to
%build appropriate initial models containing long-wavelength components for
%conventional FWI \cite[]{Ma2013, Wang2015,Chi2015, Luo2016}.
traveltime is more sensitive and linearly related to
the low-wavenumber components of the model. Therefore, traveltime inversion will be more robust and helpful to
build good initial models for conventional FWI
\cite[]{WangEtAl2014}.
\cite{Ma2013} introduced a wave equation reflected traveltime inversion method
based on dynamic image warping (DIW) to build the low wavenumber of the model. 
\cite{Chi2015} and \cite{Wang2015} employed correlation-based method to extract the
temporal or spatial lag to implement the reflection inversion. 
Elastic reflections carry the background information of the P and S wave velocities, 
which can help to rebuild the good initial velocity models for EFWI.
Unfortunately, in elastic case, traveltime shifts of a particular wave mode are difficult
to extract due to the complicated wave phenomenons, such as mode-conversions.
%It is common to observe that different mode-conversions, mainly P-P
%and P-S events, overlap and intersect with each other. The cross points between events
%would be singularities for traveltime difference estimation even using the local shift estimation 
%method like DIW.
Therefore, the estimated time shifts would be inaccurate if using the
original multicomponent seismograms directly.
%, which makes the elastic wave-equation reflection
%traveltime inversion (ERTI) difficult to implement.
In addition, since the multi-parameter trade-offs will increase the
nonlinearity of inversion, more hierarchical strategies should be considered to deal
with this problem.
\cite{WangEtAl2017} 
preconditioned the gradients of EFWI through wave mode decomposition to mitigate
parameter trade-offs and explained that this precondition partially utilize the off-diagonal Hessian blocks
when inverting $V_s$.  
As a natural tool to obtain the separated data subsets, the wave mode decomposition similarly
has the potential to precondition the elastic wave-equation reflection traveltime
inversion (ERTI) with more flexible hierarchical strategies.

%In this paper,
%we aim to tackle the traveltime misfits of P-P and P-S reflections with the help
%of wave mode decomposition and dynamic image warping (DIW) \cite[]{Hale2013}.
%Then, we use the traveltime of P-P and P-S reflections to implement the 
%WERTI method \cite[]{Ma2013} with a two-stage workflow.
%Finally, the numerical example of Sigsee2A model proves the robustness and validity of our
%elastic WERTI method.


%The key point of RWI is to project the data-domain residuals back to the reflection wavepath to 
%is to project the data-domain residuals back to the reflection wavepath to 

%for both the imaging condition and the forward modelling in each iteration.

%\cite{Zhou2015} proposed a joint FWI method to combine the diving and reflected waves to
%utilize both the conventional FWI and RWI.
%In addition, \cite{Wang:2015} found that wave mode decomposition may be beneficial to deal with the
%elastic parameter trade-offs.
%build appropriate initial models containing long-wavelength components for
%Unfortunately, in elastic case, traveltimes of a particular wave mode are difficult
%to extract due to the complicated mode-conversions. 

In this paper,   
we will exploit the traveltime misfits of PP and PS reflections to implement the ERTI
approach with the aid
of wave mode decomposition and DIW \cite[]{Hale2013}.
%In this abstract, we separate the PP and PS reflections in the observed and predicted multicomponent data to
%extract the individual traveltime residuals through DIW \cite[]{Hale2013}.
%using surface P/S wave
%separation. Dynamic image warping \cite[]{Hale2013} helps to extract the local traveltime residual
%between the observed and the predicted data.
First, 
the elastic reflection kernel and its components of different wave modes are
calculated and the implications to suppress the artifacts in the gradient calculation
are investigated.
%to investigate  the effectiveness of mode decomposition on suppressing the artifacts in the gradient calculation.
Then, P/S separation of multicomponent seismograms is applied to the observed and predicted reflection data 
to extract the individual traveltime residuals of PP and PS reflections through DIW. 
Based on the analysis of elastic reflection kernels and the separated traveltime residuals, 
we design a two-stage workflow to implement
the ERTI method,
in which the traveltime of PP reflections are firstly used to recover the background
$V_p$ model and then the traveltime of PS reflections are used to recover the background $V_s$ model
.
According to our investigation, it is quite difficult to estimate the time shifts of PS reflections if we
use the PS image to predict the corresponding reflection using demigration.
%During the $V_s$ inversion, the PS reflections are predicted by the PP image to avoid
%the wrongly estimated time shifts when using PS image.
Moreover, we precondition the $V_s$ gradient through the spatial wave mode
decomposition.
Finally, the numerical example of Sigsee2A model proves the robustness and validity of our
ERTI method.
